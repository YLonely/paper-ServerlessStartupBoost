\section{Implementation}

\pname is implemented based on containerd, runc and CRIU. 
These components have been extended to support the optimized startup of serverless functions. 
At the same time, the system also introduce the \textit{serverless function resources management component} to manage function namespace resources and checkpoint resources.

\subsection{containerd}
This system expands the original function startup process and function life cycle management logic of containerd. 
The extended containerd will use the packaged client to interact with the serverless 
function resource management component before starting the function. 
Containerd calls the service interface of resource management component passes 
in the function name and the required resource type, 
such as the namespace resource type or the checkpoint resource type. For the namespace resource type, 
it also needs to specify the detailed type of the namespace, 
such as Mount, IPC or UTS. 
After the call is successful, 
the serverless function resource management component 
will return the path and ID information of 
the corresponding resource, 
and containerd will record this information 
in the description file of the function to be started. 
The modified function life cycle management logic calls the corresponding interface of the function resource management 
component according to the resource information recorded in 
the description file after the function execution is completed, 
and returns the used resources.

\subsection{runc}

\pname has minor changes to runc. 
The extended runc component will convert the relevant resource information passed by 
containerd into a form that can be recognized by CRIU before calling CRIU to start the function process.

\subsection{CRIU}
When CRIU is started, 
it will determine whether there is the Linux namespace path information passed from runc in the startup parameters. 
If it exists, 
CRIU will directly use the setns system call to enter the namespace during the initialization of the namespace. 
At the same time, 
CRIU will determine whether the namespace type is Mount. 
If it is, CRIU also needs to setup the \textit{proxy filesystem}. 
The \textit{proxy filesystem} is implemented using FUSE\cite{fuse}, 
and CRIU mounts it to \textit{/proc} directory. 
Finally, when the CRIU initializes the memory data of the function process, 
it will calculate and record the offset position of the memory page data from the checkpoint in the virtual memory, 
and then use the mmap system call to map these pages to the virtual memory. 

\subsection{Serverless Function Resources Management Component}

As a resident service on the working node, 
this component implements the isolation environment pooling strategy and also manages the checkpoint of serverless functions. 
External services interact with this component through the packaged client library. 
There are two main back-end services of this component, 
the \textit{isolation environment resource management service} 
manages namespace resources for serverless functions 
and the \textit{checkpoint resource management service} manages
checkpoint resources of serverless functions.
When the \textit{isolation environment resource management service} starts, 
it will start multiple sub-processes based on the relevant information of the configuration file. 
These sub-processes will enter the new Linux namespace when they are created. 
The child process determines how to initialize 
the namespace by parsing the function checkpoint. 
For example, for Mount namespace, 
by parsing the checkpoint, 
the child process can obtain the type of filesystems 
that need to be mounted in the namespace. 
Subsequently, 
the child process will notify the main process of the resource management component 
to open the file descriptor corresponding to the initialized namespace 
to prevent the operating system from reclaiming the 
resources after the child process exits. 
At the same time, the component is also responsible for 
reinitializing the returned namespace resources. 
When the client requests the namespace of the function, 
the \textit{isolation environment resource management service} will return 
the path information of the corresponding namespace, 
that is, the corresponding path under the directory \textit{/proc/[pid]/fd/}. 

The \textit{function checkpoint resource management service}, 
manages the checkpoint of each function. 
it uses the containerd client to export the data of 
the function checkpoint for unified management when it is started.