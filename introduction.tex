\section{introduction}

Serverless computing is another popular trend in the current cloud computing field.
Unlike the traditional IaaS\cite{iaas} or PaaS\cite{paas},
which relies on cloud users to manage virtual machines or configure the capacity and automatic scaling of applications.
Serverless computing delegates all the work of managing virtual machines, execution environments, and software environments to the cloud provider.
Cloud providers makes all the infrastructure management operations transparent to cloud users,
and users only need to focus on writing related functions, which frees users from the trouble of managing servers.
At the same time, serverless computing provides users with a more fine-grained billing model,
which charges based on the actual amount of resources consumed when providing services.
In the Function-as-a-Service(FaaS)\cite{berkeley-view} incarnation of the serverless model,
the smallest unit of calculation used is a serverless function written by the user.
When the serverless computing platform receives a service request or a predefined event is triggered,
the serverless computing platform will initialize a temporary execution sandbox(containers or virtual machines, etc.),
and run the function uploaded by the user in the sandbox to process the request. After the execution, the sandbox will be destroyed,
and the occupied resources will be recycled for the next request.

Currently serverless computing has been supported by many platforms, including Amazon Lambda\cite{amazon}, IBM Cloud Function\cite{ibm},
Microsoft Azure Functions\cite{microsoft} and Google Cloud Functions\cite{google}.
At the same time, serverless computing has also developed rapidly in the open source community.
Serverless computing platforms represented by open source projects such as OpenFaaS\cite{openfaas}, Kubeless\cite{kubeless}, and Knative\cite{knative} have received a lot of attention.

Since serverless computing uses an event-driven approach to handle requests,
that is, when each request comes,
a new serverless function is started to process it,
which places high requirements on the efficiency of serverless functions.
The current containerized serverless functions failed to achieve the low startup
latency due to the creation of the isolation environment and the initialization
of the software environment.
The figure\ref{default-proportion} shows the "Execution and Startup/Overall" ratio of five typical serverless
computing applications which are tested on the OpenFaas platform, thoes applications run on different language runtimes.
The experiment shows that on average, the execution latency only accounted for 31\% of overall latency,
at the same time, the overall latency of containerized serverless functions is usually over 100 milliseconds(as shown in the table\ref{overall-latency}),
which obviously cannot meet the high-performance requirements of serverless computing platforms.
\begin{figure}[t]
    \centering
    \includegraphics[width=3in]{images/default-proportion.png}
    \caption{Execution(and Startup)/Overall latency ratio of serverless functions.}
    \label{default-proportion}
\end{figure}

\begin{table}[htbp]
    \centering
    \caption{The Overall latency}
    \begin{tabular}{cc}
        \hline
        Type               & Time(ms) \\ \hline
        Decompressor       & 90       \\
        Inception          & 3035     \\
        Meme               & 1228     \\
        Sentiment-analysis & 320      \\
        Search             & 2149     \\ \hline
    \end{tabular}
    \label{overall-latency}
\end{table}

Minimizing the startup latency of serverless functions is critical to improving the user experience of the platform\cite{serverless-user-experience-1,serverless-user-experience-2},
and it is also one of the major challenges facing serverless computing\cite{berkeley-view,serverless-trends}.
Existing solutions reduce the response latency of the serverless computing platform by caching function instances\cite{pool1,pool2} or customizing the virtual environment of functions\cite{firecracker,faasm},
but these solutions cannot effectively reduce the initialization latency of the application.
Subsequent research shows that application initialization accounts for a large part of the function startup latency.


This paper proposes \textbf{ProjectName}, 
a design that effectively reduces the cold start latency of containerized serverless functions. 
\textbf{ProjectName} starts a function instance by restoring it from a checkpoint image and there by skip the initialization of serverless applications. 
Since the different functions of one serverless computing application have the same isolation state, 
we can achieve the purpose of boosting the initialization of the function sandbox by caching the isolation resources. 
At the same time, because functions typically access only a small fraction of memory at runtime, 
which allows us to enable on-demand loading of memory data through mapping strategy, 
and furthermore, to share memory resources between multiple functions.


